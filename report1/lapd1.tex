\documentclass[twocolumn,twoside,10pt,a4paper]{article}

\usepackage[english]{babel}  % portuguese
\usepackage{graphicx}           % images: .png or .pdf w/ pdflatex; .eps w/ latex

%% For iso-8859-1 (latin1), comment next line and uncomment the second line
\usepackage[utf8]{inputenc}

\usepackage{times}              % PS fonts
\usepackage[T1]{fontenc}        % T1 fonts
%\usepackage{lastpage}           % to have lastpage in headers
\usepackage{url}                % urls

% geometry package
\usepackage[outer=20mm,inner=30mm,vmargin=20mm,includehead,includefoot,headheight=15pt]{geometry}

%% space between columns
\columnsep 10mm

% avoid widows and orphans
\clubpenalty=300
\widowpenalty=300

\usepackage[pdftex]{hyperref}
\hypersetup{%
    a4paper = true,              % use A4 paper 
    bookmarks = true,            % make bookmarks 
    colorlinks = true,           % false: boxed links; true: colored links
    pdffitwindow = false,        % page fit to window when opened
    pdfpagemode = UseNone,       % do not show bookmarks
    pdfpagelayout = SinglePage,  % displays a single page
    pdfpagetransition = Replace, % page transition
    linkcolor=blue,              % hyperlink colors
    urlcolor=blue,
    citecolor=blue,
    anchorcolor=green
}

\usepackage{indentfirst}       % indent also 1st paragraph

\pagestyle{myheadings}         % Option to put page headers
\markboth{{\small\it Scalable Vector Graphics}}
{{\small\it Group 08, \today}}

%\hyphenation{}                  % explicit hyphenation

% entities
\newcommand{\class}[1]{{\normalfont\slshape #1\/}}
\newcommand{\svg}{\class{SVG}}
\newcommand{\scada}{\class{SCADA}}
\newcommand{\scadadms}{\class{SCADA/DMS}}

\title{Scalable Vector Graphics}

\author{João Gradim\\
\small Faculdade de Engenharia da Universidade do Porto,\\[-0.8ex]
\small R.\ Dr.\ Roberto Frias, 4200-465 Porto\\[-0.8ex]
\small \texttt{ei05030@fe.up.pt}\\
\and
Nuno Polónia\\
\small Faculdade de Engenharia da Universidade do Porto,\\[-0.8ex]
\small R.\ Dr.\ Roberto Frias, 4200-465 Porto\\[-0.8ex]
\small \texttt{ei05037@fe.up.pt}
}

\date{\today}

\begin{document}

\maketitle
\thispagestyle{plain}

\begin{abstract}
\end{abstract}

\section{Introduction}\label{sec:intro}
Scalable Vector Graphics (SVG) is an XML markup language specification for describing two-dimensional vector graphics.\\
W3C started developing it in 1999 after Adobe and Microsoft submitted two different and competing standards, the Precision Graphic Markup Language (PGML) and the Vector Markup Language (VML) respectively, both in 1998.\\


\subsection{Vector and Raster Graphics}
Vector graphics rely on the use of mathematical equations to describe primitives which are used to construct images. Being 


\section{XML Vector image formats}
\subsection{Vector Image Formats}

%http://en.wikipedia.org/wiki/List_of_vector_graphics_markup_languages
\subsubsection{Precision Graphics Markup Language (PGML)}

PGML first appeared in 1998, as an initial draft proposed to W3C, despite of not having been adopted as a recommendation\cite{w3c:pgml}. PGML is a 2D XML-based vector drawing language and uses the imaging model of the PostScript language and PDF as its basis in order to provide simple-to-use graphical objects and precise visual fidelity.

\subsubsection{Vector Markup Language (VML)}

As with PGML, VML was submitted as a proposed standard to the W3C in 1998 by Autodesk, Hewlett-Packard, Macromedia, Microsoft, and Visio.

VML was designed to support the markup of vector graphic information in the same way that HTML supports the markup of textual information. Within VML the content is composed of paths described using connected lines and curves. The markup provides both semantic and presentation information about these paths.

VML is mainly used by Microsoft and its related products, such as Microsoft Word and Internet Explorer.

\section{Model Description}

SVG is a markup language used to describe two-dimensional graphics and graphical applications. SVG is a royalty-free, vendor-neutral open standard developed under the W3C (World Wide Web Consortium) Process, with a strong development support by Adobe and other major companies.
Being an XML language, SVG benefits from techniques and technologies applicable to other XML documents, such as XSLT transformations, embedding an SVG document in another XML documents using namespaces, and even styling through CSS stylesheets.

\subsection{Elements}

SVG uses both self-closing and content elements. For example, a \verb!<rect>! element is defined solely based on its attributes, such as \verb!width! and \verb!height!, and does not need any content to describe it. However, a series of \verb!<rect>! elements may be part of a \verb!<g>! (group) element. This is because a group element only makes sense when it groups a series of elements as its content.

%http://www.w3.org/TR/SVGTiny12/types.html
%http://www.w3.org/TR/SVGTiny12/struct.html

\subsection{Attributes}

Attributes are a big part of the SVG specification. As SVG is a language that describes a drawing (composed by a series of graphical elements, such as rectangles, ellipses, and others) as an XML document, most of its elements are described by their attributes

\section{Schema}

As SVG does not have a DTD, it is not necessary to specify a DOCTYPE. There is, however, a schema available for validation based on the RelaxNG schema\footnote{RelaxNG is a namespace-aware schema language that uses the datatypes from XML Schema Part 2. This allows namespaces and modularity to be much more naturally expressed than using tradicional DTD syntax.}

\section{Final Remarks}

SVG is a heavily documented, greatly supported markup language for representing vectorial images. It is extremely versatile, being able to be used for both on-screen and print with losing any quality.

%% auto bibliographic list 
\renewcommand{\bibname}{Referências}
\bibliographystyle{unsrt-pt}
\bibliography{lapd1}

\end{document}


