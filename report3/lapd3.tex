\documentclass[twocolumn,twoside,10pt,a4paper]{article}

\usepackage[english]{babel}  % portuguese
\usepackage{graphicx}           % images: .png or .pdf w/ pdflatex; .eps w/ latex

%% For iso-8859-1 (latin1), comment next line and uncomment the second line
\usepackage[utf8]{inputenc}

\usepackage{times}              % PS fonts
\usepackage[T1]{fontenc}        % T1 fonts
%\usepackage{lastpage}           % to have lastpage in headers
\usepackage{url}                % urls

% geometry package
\usepackage[outer=20mm,inner=30mm,vmargin=20mm,includehead,includefoot,headheight=15pt]{geometry}

%% space between columns
\columnsep 10mm

% avoid widows and orphans
\clubpenalty=300
\widowpenalty=300

\usepackage{listings}
\lstdefinestyle{xml}{
	language=XML,
	extendedchars=true,
	inputencoding=latin1,
	tabsize=4,
	showstringspaces=false,
	basicstyle=\scriptsize,
	keywordstyle=\ttfamily\color{blue},
	stringstyle=\ttfamily\color{orange},
	identifierstyle=\ttfamily,
	commentstyle=\ttfamily\color{darkred},
	morecomment=[s][\ttfamily\color{javadoc}]{/**}{*/},
	numbers=left
}

\usepackage[pdftex]{hyperref}
\hypersetup{%
    a4paper = true,              % use A4 paper
    bookmarks = true,            % make bookmarks
    colorlinks = true,           % false: boxed links; true: colored links
    pdffitwindow = false,        % page fit to window when opened
    pdfpagemode = UseNone,       % do not show bookmarks
    pdfpagelayout = SinglePage,  % displays a single page
    pdfpagetransition = Replace, % page transition
    linkcolor=blue,              % hyperlink colors
    urlcolor=blue,
    citecolor=blue,
    anchorcolor=green
}

\usepackage{indentfirst}       % indent also 1st paragraph

\pagestyle{myheadings}         % Option to put page headers
\markboth{{\small\it xml2json}}
{{\small\it Group 08, \today}}

%\hyphenation{}                  % explicit hyphenation

% entities
\newcommand{\class}[1]{{\normalfont\slshape #1\/}}

\title{xml2json}

\author{João Gradim\\
\small Faculdade de Engenharia da Universidade do Porto,\\[-0.8ex]
\small R.\ Dr.\ Roberto Frias, 4200-465 Porto\\[-0.8ex]
\small \texttt{ei05030@fe.up.pt}\\
\and
Nuno Polónia\\
\small Faculdade de Engenharia da Universidade do Porto,\\[-0.8ex]
\small R.\ Dr.\ Roberto Frias, 4200-465 Porto\\[-0.8ex]
\small \texttt{ei05037@fe.up.pt}
}

\date{\today}

\begin{document}

\maketitle
\thispagestyle{plain}

\begin{abstract}

\end{abstract}

\section{Introduction}\label{sec:intro}
On the newly programmable Web, mashups are flourishing \cite{maximilien}

\section{Application}\label{sec:application}

\subsection{User Requirements}\label{sec:user-requirements}

\begin{itemize}
    \item The user should be able to transform any XML document to a valid JSON representation
    \item The user should be able to transform XML to JSON visually (in a compatible browser) or programmatically (through and HTTP GET request)
    \item The system should provide a live XML to JSON editor, in order to preview the result of a transformation
\end{itemize}

\subsection{Similar Projects}\label{sec:similar-projects}

Most modern programming languages (such as Ruby and Python) provide support for both XML and JSON formats, either built-in or with 3rd-party libraries, and allow programmers to easily convert between these representations. However, there is not an universal, language-agnostic service that allows programmers to easily convert an XML document to a JSON representation, that ensures that the transformation does not change whichever the programming language used.

\subsection{Architecture}\label{sec:architecture}

As previously stated, xml2json acts primarily as a proxy that translates an XML document to a JSON representation. As such, a user may perform a GET request to the xml2json api converter url \footnote{at the moment of writing, \url{http://xml2json.ath.cx/convert.json}}, specifying the \verb!request-url! parameter --- which corresponds to the location of the XML document they wish to convert to JSON. The application then makes a request to whichever address the user specified, retrieves the XML document, converts it to JSON and finally responds to the user with the JSON representation of the specified XML document. While performing the transformation, the system also saves the access to the database in order to keep track of which APIs have been accessed.

Figure \ref{fig:system_arch} represents this architecture.

\begin{figure}[h]
    \centering
    \includegraphics[width=80mm]{images/arch.png}
    \caption{System Architecture}
    \label{fig:system_arch}
\end{figure}

\subsection{Implementation}\label{sec:implementation}

The application was implemented using the Ruby programming language and the Ruby on Rails framework for the backend, and HTML5, CSS 2/3 and JavaScript for the frontend.

\section{Results}\label{sec:results}

\section{Final Remarks}\label{sec:final-remarks}

%% auto bibliographic list
\renewcommand{\bibname}{References}
\bibliographystyle{unsrt-pt}
\bibliography{lapd3}

\end{document}

